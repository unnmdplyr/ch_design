

\section{System design}

This section describes the design of the whole system. Visualizes the control
and dataflow between the subsystems. Explains interfaces, dependencies between
subsystems. Input, output to and from each subsystem.

Full description is created by \citeauthor{profcpp11} at page 53 in chapter
``What is Programming Design?''.

The \textbf{subsystems} are the followings:

\begin{description}
	\item	[Vertex provider] Stores the vertex related data which comes as
		input into the algorithm and gives back the vertices to the client code
		as it requests.

	\item	[Tetrahedron vertices chooser]	Chooses from the input 4 vertices
		which will make the initial tetrahedron. See Algorithm \ref{convhull}
		Line \ref{tetrah}.
		This subsystem will check whether the input is valid and will give back
		an error result if it finds that not. See Algorithm \ref{convhull} Line
		\ref{input_check}.

	\item	[Randomizer] Randomizes the input points. See Algorithm
				\ref{convhull} Line \ref{random}.

	\item	[Doubly-connected edge list.] With the help of this data
				representation, the faces of the polytope can be stored.

	\item	[Conflict graph.] Stores the information if a new vertex is added to
				the polytope then which faces must be removed.
				
	\item	[Face indices provider] At generating the output this will convert
		the face indices of the polytope into the required format according
		to the interface specification. See Section \ref{sec:interface}.

\end{description}
