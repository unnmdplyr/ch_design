

\subsection{Doubly-connected edge list}

This subsection deals with the doubly-connected edge list related realization
part.

\textbf{Requirements}:

\begin{itemize}

\item	We will keep the convention that the half-edges are directed such that
			the ones bounding any face form a \textbf{counterclockwise} cycle
			when seen from the outside of the polytope. This requirement
			originates from \citeauthor{compgeo08} page 247.

\item	Always will exist a face which represents the \textit{horizon} of $p_r$
			on \convhull($P_{r-1}$). But this doesn't mean that cannot be states
			where no any edges will belong to this face. E.g. when the polytope
			is closed and doesn't miss any faces from the polytope.

\end{itemize}


\textbf{Required operations}:

\begin{description}

	\item	[Add new face] to existing faces. See Algorithm \ref{alg:add_face}
		Line \ref{alg:line:add_face_func}.
	
	
	\item	[Remove existing face] from the polytope.
		\begin{itemize}
			\item	At determining the horizon of a point on the given polytope,
				the face must remain in the polytope whose infinite plane
				includes the given point.
		\end{itemize}
		
	\item	[Merging coplanar faces] in the polytope which are adjacent to each
				other.

\end{description}


\newcommand{\vara}		{$\mathcal{V}_a$}
\newcommand{\lste}		{$\mathcal{L}_e$}

	\begin{algorithm}[H]
	\caption{Add a new face algorithm}						\label{alg:add_face}
	\begin{algorithmic}[1]
		\Require A set $P_m$ of $m$ points in three-space where
			$P_m \subseteq P$ and $m \leq n$.
		\Ensure The \textit{id} of the recently added face.

		\Statex
		\Function{AddFace}{ $P_m$ }				\label{alg:line:add_face_func}

			\State Create a new \texttt{IncidentFace} ID.
			\Statex								\Comment{The edges of the face}

			\ForAll { $e \in \{e_{i,i+1}\} \cup \{e_{m,0}\}$,
													where $0 \leq i \leq m-1$ }
				\If { $e$ already $\exists$ }
					\State \vara $\gets$ \texttt{IncidentFace} ID of $e$
						
				\Else
					\State Add $e$ to the face.
				\EndIf

				\State Initialize its \texttt{next} and \texttt{prev} members.
				\State \texttt{IncidentFace} of $e \gets$ the recently created
					face ID
			\EndFor

			\Statex
			
			\State $allTwinsExist \gets true $
			\State $noTwinsExist   \gets true $

			\ForAll { $e \in \{e_{i,i-1}\} \cup \{e_{0,m}\}$,
													where $1 \leq i \leq m$ }
				\Comment{The twin edges}
				
				\If { $e$ $\nexists$ }
					\State Add $e$ to the polytope.
					\State \texttt{IncidentFace} of $e \gets$ \texttt{null} face
						ID
					\State Put $e$ into list \lste.
					\State $allTwinsExist \gets false$
				\Else
					\State $noTwinsExist   \gets false $
					\State \texttt{IncidentFace} ID $\gets$ the
						\texttt{IncidentFace} ID of $e$
				\EndIf
			\EndFor

			\Statex

			\If { $allTwinsExist == true$ }
				\Comment {All twin edges already existed}
				\State Remove the inner face with ID \texttt{IncidentFace} ID.
			\EndIf

			\Statex

			\If { $noTwinsExist == true$ }
				\Comment {No twin edges existed yet}
				\State	Create new face ID and new face $f$.
				\State	\vara $\gets$ face ID of $f$
				\State	\texttt{OuterComponent} of $f \gets $ \texttt{null} edge
					ID
				\State	\texttt{InnerComponent} of $f \gets $ the first edge
					from the list \lste
			\EndIf

		\algstore{bkbreak}
	\end{algorithmic}
	\end{algorithm}

	\begin{algorithm}[h]
	\caption{Add a new face algorithm - Part 2}
	\begin{algorithmic}[1]
		\algrestore{bkbreak}
			\Statex

			\ForAll { $e \in$ \lste }
				\Comment{Building the inner face}
				\State Make a round trip around the inner face at the following
					manner.
				\Statex

				\If {this edge is the first in the list \lste}
					\State \texttt{IncidentFace} ID $\gets$ \vara
				\Else
					\State Create new \texttt{IncidentFace} ID.
				\EndIf
				
				\Statex

				\State Initialize \texttt{next} and \texttt{prev} members of $e$
					by looking for edges which
				\Statex[2]	point out from the vertex which is the endpoint of
					$e$ and its
				\Statex[2]	\texttt{Inci\-dent\-Face} member is \texttt{null}
					face ID or ID stored in \vara.
				\Statex[2]	Write over its \texttt{IncidentFace} ID member to
					the current
				\Statex[2]	\texttt{IncidentFace} ID.
				\Statex

				\ForAll {edges which is touched during the round trip}
					\If {edge.\texttt{IncidentFace} ID $==$ \texttt{null}
						face ID}
						\State Remove this edge ID from the list \lste.
					\EndIf
				\EndFor
			\EndFor
			\Statex


		\EndFunction
	\end{algorithmic}
	\end{algorithm}

	
	

	%	Get the next edge. And begin to create with it the other
	%	\texttt{InnerComponent}s.





