

\section{The algorithm}

The whole algorithm is the following:

\newcommand{\conv}		{$\mathcal{C}$}
\newcommand{\convhull}	{$\mathcal{CH}$}
\newcommand{\graph}		{$\mathcal{G}$}
\newcommand{\conf}		{$F_{conflict}(p_r)$}
\newcommand{\lst}		{$\mathcal{L}$}

	\begin{algorithm}[H]
	\caption{ConvexHull algorithm}								\label{convhull}
	\begin{algorithmic}[1]
		\Require A set $P$ of $n$ points in three-space.
		\Ensure The convex hull \convhull($P$) of $P$.

		\Statex
		\Function{ConvexHull}{ $P$ }

			\State Find four points $p_1, p_2, p_3, p_4$ in $P$ that form
						a tetrahedron.						\label{input_check}

			\State \conv $ \gets $ \convhull (\{$p_1,p_2,p_3,p_4$\})
															\label{tetrah}

			\State Compute a random permutation $p_5, p_6, \dots, p_n$ of
						the remaining points.				\label{random}

			\State Initialize the conflict graph \graph with all visible pairs
					($p_t, f$), where $f$ is a
			\Statex[1] facet of \conv and $t \textgreater 4$.
			\OldStatex

			\For {$r \gets 5$ \textbf{to} $n$}
				%\State (* Insert $p_r$ into \conv: *)
				\Comment{Insert $p_r$ into \conv}
				\If {\conf  is not empty %(* that is, $p_r$ lies outside \conv *)
					}
					\Comment{that is, $p_r$ lies outside \conv}
					\State Delete all facets in \conf from \conv.
					\State Walk along the boundary of the visible region of
								$p_r$ (which
					\Statex[3]	consists exactly of the facets in \conf) and
								create a list \lst of
					\Statex[3]	horizon edges in order.
					\OldStatex
								
					\ForAll{$e \in$ \lst}
						\State Connect $e$ to $p_r$ by creating a triangular
									facet $f$.
						\If {$f$ is coplanar with its neighbor facet $f'$ along
								$e$}
							\State Merge $f$ and $f'$ into one facet, whose
									conflict list is the
							\Statex[5]	same as that of $f'$.
						\Else %(* Determine conflicts for $f$: *)
							\Comment{Determine conflicts for $f$}
							\State Create a node for $f$ in \graph.
							\State Let $f_1$ and $f_2$ be the facets incident to
									$e$ in the old convex
							\Statex[5]	hull.
							\State $P(e) \gets P_{conflict}(f_1)
										\cup P_{conflict}(f_2)$
							\ForAll {points $p \in P(e)$}
								\State If $f$ is visible from $p$, add $(p,f)$
										to \graph.
							\EndFor
						\EndIf
					\EndFor
					\State Delete the node corresponding to $p_r$ and the nodes
					\Statex[3]	corresponding to the facets in \conf from
									\graph, together with
					\Statex[3]	their incident arcs.
				\EndIf
			\EndFor

			\State \Return \conv
		\EndFunction
	\end{algorithmic}
	\end{algorithm}


The algorithm is provided by \citeauthor{compgeo08} at page 249.

The whole concept is the following. We have at the beginning a
\textbf{tetrahed\-ron} which has 4 vertices. In every step will be \textbf{add}ed
to this polytope a new \textbf{vertex}. If the new vertex is \textbf{inside} the
polytope, then ignore it. Else create a new polytope where the new vertex is
part of it.

%%------------------------------------------------------------------------------

\section{Requirements}

This section enumerates the possible \textbf{requirements}.

\begin{itemize}

	\item	A \textbf{library} must be created or compiled which establishes the
		given algorithm.

	\item	The library provides an \textbf{interface}. The user of the library
		can reach the functionality (provided by the library) through this
		interface (or API).
		
	\item	The vertex data mustn't be changed outside the library until the
		process of the convex hull determination runs.

\end{itemize}

%%------------------------------------------------------------------------------

\section{Library interface}
\label{sec:interface}

This section explains the interface of the library. Only one function can be
called. This is the function \texttt{ch::ConvexHull()}.


\lstset{language=[GNU]C++,
		morekeywords={[2]size_t,T,vector,res_t},
		keywordstyle={[2]\color{cyan}\bfseries}}
\begin{lstlisting}[caption={The only interface function},
					label={lst:intf_func}]
///	\return	With the result which indicates whether does any
///					solutions exist for the given input.
template <typename T>
ch::res_t ConvexHull(
						const T* const ptr,
						const std::size_t coordPad,
						const std::size_t diffSize,
						const std::size_t vertNum,
						std::vector< std::vector<std::size_t> >& indices
						);
\end{lstlisting}

Parameters which are shown in the Listing \ref{lst:intf_func} are the following:

\bigskip

\begin{tabularx}{1\textwidth}{
				l
				>{\setlength\hsize{1\hsize}\raggedright\arraybackslash}X }
\textbf{Parameter} & \textbf{Meaning of the parameter}				\\
\hline
ptr			& pointer to the first element							\\
coordPad	& padding between coordinates in bytes					\\
diffSize	& difference between begin address of
							two consecutive vertices in bytes		\\
vertNum		& number of vertices									\\
indices		& the ouput parameter. Contains the face indices of the
									result convex hull.				\\
\hline
\end{tabularx}
	
%%------------------------------------------------------------------------------




