


\section{Subsystem design}

This section describes the details of each subsystem. Classes, class
hierarchies, data structures, algorithms, threading model, error handling and
possible additional aspects.


\textbf{Common requirements for all subsystems:}

\begin{description}
	\item	[\newReqCommon{standalonedir}]	In the source code all subsystems
			must have a stand alone directory independent from the other ones.
			This means that every subsystem must have an appropriate interface
			towards the other subsystems. At this way the testing for all
			subsystems must be easier. So ``only'' a (sub)interface must be
			tested.
	
	\item	[\newReqCommon{reachingfunctionality}] But each subsystem must be
			able to reach the functionality of the other subsystems which they
			provide.

	\item	[\newReqCommon{linktoone}] Each subsystem compile its own binary
			files and these binary files can be linked into one executable file.
	
\end{description}

%get req common: \getReqCommon{standalonedir}

%-------------------------------------------------------------------------------
%	Vertex provider
%-------------------------------------------------------------------------------

\subsection{Vertex provider subsystem}


This subsystem stores the vertex related data which comes as input into the
algorithm.

The class diagram for vertex provider can be seen on
the Figure \ref{fig:vertexprovider}.

\begin{figure}[h]
	\centering
	\includegraphics[width=0.5\textwidth]{VertexProvider.pdf}
	\caption{Class diagram for vertex provider}
	\label{fig:vertexprovider}
\end{figure}


\textbf{Requirements}:

\begin{itemize}
	\item	Receiving the input:
		\begin{itemize}
			\item	Copying the input. This is not efficient.
			\item	Compute where is each coordinate if given the address of the
						first element, the size of one coordinate, the padding
						value between the coordinates, the padding between
						the vertices and the number of vertices.
						In this case the randomization how should be solved?
						Randomization won't be dependent on this data
						representation. The randomizer subsystem only requires
						the number of vertices from this subsystem. And the
						vertices the initial tetrahedron build up from.
		\end{itemize}
	\item	The functionality of this subsystem must be reachable from the other
		subsystems as well.
\end{itemize}


%-------------------------------------------------------------------------------
%	Tetrahedron vertices chooser
%-------------------------------------------------------------------------------

\subsection{Tetrahedron vertices chooser subsystem}

The class diagram for choosing the vertices to the tetrahedron can be seen on
the Figure \ref{fig:tetrahedronvertices}.

\begin{figure}[h]
	\centering
	\includegraphics[width=0.7\textwidth]{TetrahedronVertices.pdf}
	\caption{Class diagram for choosing vertices to the tetrahedron}
	\label{fig:tetrahedronvertices}
\end{figure}


Possible code for \texttt{VertexChooser::ChooseVertices()}  can be seen in
the Listing \ref{lst:vertexchooser}.

\lstset{language=[GNU]C++,
		morekeywords={[2]size_t,T,vector,res_t,CH_SUCCESS,vid_t,
						VertexChooser,VertexChooserRule},
		keywordstyle={[2]\color{cyan}\bfseries}}
\begin{lstlisting}[caption={Possible code for the
							\texttt{VertexChooser::ChooseVertices()} function},
					label={lst:vertexchooser}]
ch::res_t VertexChooser::ChooseVertices(
									const std::vector<VertexChooserRule> &rule,
									std::vector<vid_t> &point )
{
	for ( std::size_t i=0; i < rule.size(); ++i )
	{
		ch::res_t res = rule[i].ChooseVertex( dh, point );
		if ( res != CH_SUCCESS )
			return res;
	}
	return CH_SUCCESS;
}
\end{lstlisting}


Possible code for calling the \texttt{VertexChooser::ChooseVertices()} function
can be seen in the Listing \ref{lst:call_vertexchooser}.

\lstset{language=[GNU]C++,
		morekeywords={[2]vid_t,vector,res_t,CH_SUCCESS,Collection,
						VertexChooserRule1st, VertexChooserRule4th,
						TetrahedronVerticesChooser, VertexChooserRule,
						VertexChooser},
		keywordstyle={[2]\color{cyan}\bfseries}}
\begin{lstlisting}[caption={Possible code for the calling the
				\texttt{VertexChooser::ChooseVertices()} function},
					label={lst:call_vertexchooser}]
Collection<VertexChooserRule> collection;
collection.push_back( new VertexChooserRule1st );
//	...
collection.push_back( new VertexChooserRule4th );

std::vector<vid_t> point;

VertexChooser &vc = TetrahedronVerticesChooser(dh);

ch::res_t res = vc.ChooseVertices( collection, point );
if ( res != CH_SUCCESS )
	return res;
\end{lstlisting}


%-------------------------------------------------------------------------------
%	Randomizer
%-------------------------------------------------------------------------------

\subsection{Randomizer subsystem}

\colorbox{red}{How to create an abstraction? \texttt{Iterator}? Inner or outer?}

If the \texttt{Randomizer} is realized as on-air randomizer then it can be
realized as \texttt{RandomIterator}.

In its constructor receives the points $P$ and the index of the points in $P$
which forms the tetrahedron. So these points (which form the tetrahedron) won't
be enumerated by the Iterator.

\subsubsection{Realization}

The \texttt{RandomIterator} must contain a vector whose every element will be
initialized to \texttt{-1} except the ones which form the tetrahedron. They will
contain e.g. 0 or 1. The \texttt{RandomIterator} always chooses an index which
is not \texttt{-1}. 







